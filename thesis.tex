\documentclass[en]{pracamgr}

% ======== Custom package setup ========
\usepackage{amsthm}
\usepackage[dvipsnames]{xcolor}
\usepackage{hyperref}
\usepackage{listings}
\usepackage{float}
\hypersetup{
    colorlinks=true,
    linkcolor=blue,
    filecolor=magenta,      
    urlcolor=RoyalBlue,
    pdfpagemode=FullScreen,
}

\definecolor{dkgreen}{rgb}{0,0.6,0}
\definecolor{gray}{rgb}{0.5,0.5,0.5}
\definecolor{mauve}{rgb}{0.58,0,0.82}
\lstset{frame=tb,
    language=Haskell,
    aboveskip=3mm,
    belowskip=3mm,
    showstringspaces=false,
    columns=flexible,
    basicstyle={\small\ttfamily},
    numbers=none,
    numberstyle=\tiny\color{gray},
    keywordstyle=\color{blue},
    commentstyle=\color{dkgreen},
    stringstyle=\color{mauve},
    breaklines=false,
    breakatwhitespace=true,
    frame=none,
    escapeinside={(*}{*)},          % if you want to add LaTeX within your code
    tabsize=4
}
\lstdefinelanguage{JavaScript}{
  keywords={typeof, new, true, false, catch, function, return, null, catch, switch, var, if, in, while, do, else, case, break, class, this},
  keywordstyle=\color{blue}\bfseries,
  ndkeywords={export, boolean, throw, implements, import, string, number, constructor},
  ndkeywordstyle=\color{darkgray}\bfseries,
  identifierstyle=\color{black},
  sensitive=false,
  comment=[l]{//},
  morecomment=[s]{/*}{*/},
  commentstyle=\color{purple}\ttfamily,
  stringstyle=\color{red}\ttfamily,
  morestring=[b]',
  morestring=[b]"
}
\lstdefinelanguage{Kotlin}{
  comment=[l]{//},
  commentstyle={\color{gray}\ttfamily},
  emph={filter, first, firstOrNull, forEach, lazy, map, mapNotNull, println},
  emphstyle={\color{OrangeRed}},
  identifierstyle=\color{black},
  keywords={!in, !is, abstract, actual, annotation, as, as?, break, by, catch, class, companion, const, constructor, continue, crossinline, data, delegate, do, dynamic, else, enum, expect, external, false, field, file, final, finally, for, fun, get, if, import, in, infix, init, inline, inner, interface, internal, is, lateinit, noinline, null, object, open, operator, out, override, package, param, private, property, protected, public, receiveris, reified, return, return@, sealed, set, setparam, super, suspend, tailrec, this, throw, true, try, typealias, typeof, val, var, vararg, when, where, while},
  keywordstyle={\color{NavyBlue}\bfseries},
  morecomment=[s]{/*}{*/},
  morestring=[b]",
  morestring=[s]{"""*}{*"""},
  ndkeywords={@Deprecated, @JvmField, @JvmName, @JvmOverloads, @JvmStatic, @JvmSynthetic, Array, Byte, Double, Float, Int, Integer, Iterable, Long, Runnable, Short, String, Any, Unit, Nothing},
  ndkeywordstyle={\color{BurntOrange}\bfseries},
  sensitive=true,
  stringstyle={\color{ForestGreen}\ttfamily},
}
% rust listings from https://github.com/denki/listings-rust
\lstdefinelanguage{Rust}{%
  sensitive%
, morecomment=[l]{//}%
, morecomment=[s]{/*}{*/}%
, moredelim=[s][{\itshape\color[rgb]{0,0,0.75}}]{\#[}{]}%
, morestring=[b]{"}%
, alsodigit={}%
, alsoother={}%
, alsoletter={!}%
%
%
% [1] reserve keywords
% [2] traits
% [3] primitive types
% [4] type and value constructors
% [5] identifier
%
, morekeywords={break, continue, else, for, if, in, loop, match, return, while}  % control flow keywords
, morekeywords={as, const, let, move, mut, ref, static}  % in the context of variables
, morekeywords={dyn, enum, fn, impl, Self, self, struct, trait, type, union, use, where}  % in the context of declarations
, morekeywords={crate, extern, mod, pub, super}  % in the context of modularisation
, morekeywords={unsafe}  % markers
, morekeywords={abstract, alignof, become, box, do, final, macro, offsetof, override, priv, proc, pure, sizeof, typeof, unsized, virtual, yield}  % reserved identifiers
%
% grep 'pub trait [A-Za-z][A-Za-z0-9]*' -r . | sed 's/^.*pub trait \([A-Za-z][A-Za-z0-9]*\).*/\1/g' | sort -u | tr '\n' ',' | sed 's/^\(.*\),$/{\1}\n/g' | sed 's/,/, /g'
, morekeywords=[2]{Add, AddAssign, Any, AsciiExt, AsInner, AsInnerMut, AsMut, AsRawFd, AsRawHandle, AsRawSocket, AsRef, Binary, BitAnd, BitAndAssign, Bitor, BitOr, BitOrAssign, BitXor, BitXorAssign, Borrow, BorrowMut, Boxed, BoxPlace, BufRead, BuildHasher, CastInto, CharExt, Clone, CoerceUnsized, CommandExt, Copy, Debug, DecodableFloat, Default, Deref, DerefMut, DirBuilderExt, DirEntryExt, Display, Div, DivAssign, DoubleEndedIterator, DoubleEndedSearcher, Drop, EnvKey, Eq, Error, ExactSizeIterator, ExitStatusExt, Extend, FileExt, FileTypeExt, Float, Fn, FnBox, FnMut, FnOnce, Freeze, From, FromInner, FromIterator, FromRawFd, FromRawHandle, FromRawSocket, FromStr, FullOps, FusedIterator, Generator, Hash, Hasher, Index, IndexMut, InPlace, Int, Into, IntoCow, IntoInner, IntoIterator, IntoRawFd, IntoRawHandle, IntoRawSocket, IsMinusOne, IsZero, Iterator, JoinHandleExt, LargeInt, LowerExp, LowerHex, MetadataExt, Mul, MulAssign, Neg, Not, Octal, OpenOptionsExt, Ord, OsStrExt, OsStringExt, Packet, PartialEq, PartialOrd, Pattern, PermissionsExt, Place, Placer, Pointer, Product, Put, RangeArgument, RawFloat, Read, Rem, RemAssign, Seek, Shl, ShlAssign, Shr, ShrAssign, Sized, SliceConcatExt, SliceExt, SliceIndex, Stats, Step, StrExt, Sub, SubAssign, Sum, Sync, TDynBenchFn, Terminal, Termination, ToOwned, ToSocketAddrs, ToString, Try, TryFrom, TryInto, UnicodeStr, Unsize, UpperExp, UpperHex, WideInt, Write}
, morekeywords=[2]{Send}  % additional traits
%
, morekeywords=[3]{bool, char, f32, f64, i8, i16, i32, i64, isize, str, u8, u16, u32, u64, unit, usize, i128, u128}  % primitive types
%
, morekeywords=[4]{Err, false, None, Ok, Some, true}  % prelude value constructors
% grep 'pub \(type\|struct\|enum\) [A-Za-z][A-Za-z0-9]*' -r . | sed 's/^.*pub \(type\|struct\|enum\) \([A-Za-z][A-Za-z0-9]*\).*/\2/g' | sort -u | tr '\n' ',' | sed 's/^\(.*\),$/{\1}\n/g' | sed 's/,/, /g'    
, morekeywords=[3]{AccessError, Adddf3, AddI128, AddoI128, AddoU128, ADDRESS, ADDRESS64, addrinfo, ADDRINFOA, AddrParseError, Addsf3, AddU128, advice, aiocb, Alignment, AllocErr, AnonPipe, Answer, Arc, Args, ArgsInnerDebug, ArgsOs, Argument, Arguments, ArgumentV1, Ashldi3, Ashlti3, Ashrdi3, Ashrti3, AssertParamIsClone, AssertParamIsCopy, AssertParamIsEq, AssertUnwindSafe, AtomicBool, AtomicPtr, Attr, auxtype, auxv, BackPlace, BacktraceContext, Barrier, BarrierWaitResult, Bencher, BenchMode, BenchSamples, BinaryHeap, BinaryHeapPlace, blkcnt, blkcnt64, blksize, BOOL, boolean, BOOLEAN, BoolTrie, BorrowError, BorrowMutError, Bound, Box, bpf, BTreeMap, BTreeSet, Bucket, BucketState, Buf, BufReader, BufWriter, Builder, BuildHasherDefault, BY, BYTE, Bytes, CannotReallocInPlace, cc, Cell, Chain, CHAR, CharIndices, CharPredicateSearcher, Chars, CharSearcher, CharsError, CharSliceSearcher, CharTryFromError, Child, ChildPipes, ChildStderr, ChildStdin, ChildStdio, ChildStdout, Chunks, ChunksMut, ciovec, clock, clockid, Cloned, cmsgcred, cmsghdr, CodePoint, Color, ColorConfig, Command, CommandEnv, Component, Components, CONDITION, condvar, Condvar, CONSOLE, CONTEXT, Count, Cow, cpu, CRITICAL, CStr, CString, CStringArray, Cursor, Cycle, CycleIter, daddr, DebugList, DebugMap, DebugSet, DebugStruct, DebugTuple, Decimal, Decoded, DecodeUtf16, DecodeUtf16Error, DecodeUtf8, DefaultEnvKey, DefaultHasher, dev, device, Difference, Digit32, DIR, DirBuilder, dircookie, dirent, dirent64, DirEntry, Discriminant, DISPATCHER, Display, Divdf3, Divdi3, Divmoddi4, Divmodsi4, Divsf3, Divsi3, Divti3, dl, Dl, Dlmalloc, Dns, DnsAnswer, DnsQuery, dqblk, Drain, DrainFilter, Dtor, Duration, DwarfReader, DWORD, DWORDLONG, DynamicLibrary, Edge, EHAction, EHContext, Elf32, Elf64, Empty, EmptyBucket, EncodeUtf16, EncodeWide, Entry, EntryPlace, Enumerate, Env, epoll, errno, Error, ErrorKind, EscapeDebug, EscapeDefault, EscapeUnicode, event, Event, eventrwflags, eventtype, ExactChunks, ExactChunksMut, EXCEPTION, Excess, ExchangeHeapSingleton, exit, exitcode, ExitStatus, Failure, fd, fdflags, fdsflags, fdstat, ff, fflags, File, FILE, FileAttr, filedelta, FileDesc, FilePermissions, filesize, filestat, FILETIME, filetype, FileType, Filter, FilterMap, Fixdfdi, Fixdfsi, Fixdfti, Fixsfdi, Fixsfsi, Fixsfti, Fixunsdfdi, Fixunsdfsi, Fixunsdfti, Fixunssfdi, Fixunssfsi, Fixunssfti, Flag, FlatMap, Floatdidf, FLOATING, Floatsidf, Floatsisf, Floattidf, Floattisf, Floatundidf, Floatunsidf, Floatunsisf, Floatuntidf, Floatuntisf, flock, ForceResult, FormatSpec, Formatted, Formatter, Fp, FpCategory, fpos, fpos64, fpreg, fpregset, FPUControlWord, Frame, FromBytesWithNulError, FromUtf16Error, FromUtf8Error, FrontPlace, fsblkcnt, fsfilcnt, fsflags, fsid, fstore, fsword, FullBucket, FullBucketMut, FullDecoded, Fuse, GapThenFull, GeneratorState, gid, glob, glob64, GlobalDlmalloc, greg, group, GROUP, Guard, GUID, Handle, HANDLE, Handler, HashMap, HashSet, Heap, HINSTANCE, HMODULE, hostent, HRESULT, id, idtype, if, ifaddrs, IMAGEHLP, Immut, in, in6, Incoming, Infallible, Initializer, ino, ino64, inode, input, InsertResult, Inspect, Instant, int16, int32, int64, int8, integer, IntermediateBox, Internal, Intersection, intmax, IntoInnerError, IntoIter, IntoStringError, intptr, InvalidSequence, iovec, ip, IpAddr, ipc, Ipv4Addr, ipv6, Ipv6Addr, Ipv6MulticastScope, Iter, IterMut, itimerspec, itimerval, jail, JoinHandle, JoinPathsError, KDHELP64, kevent, kevent64, key, Key, Keys, KV, l4, LARGE, lastlog, launchpad, Layout, Lazy, lconv, Leaf, LeafOrInternal, Lines, LinesAny, LineWriter, linger, linkcount, LinkedList, load, locale, LocalKey, LocalKeyState, Location, lock, LockResult, loff, LONG, lookup, lookupflags, LookupHost, LPBOOL, LPBY, LPBYTE, LPCSTR, LPCVOID, LPCWSTR, LPDWORD, LPFILETIME, LPHANDLE, LPOVERLAPPED, LPPROCESS, LPPROGRESS, LPSECURITY, LPSTARTUPINFO, LPSTR, LPVOID, LPWCH, LPWIN32, LPWSADATA, LPWSAPROTOCOL, LPWSTR, Lshrdi3, Lshrti3, lwpid, M128A, mach, major, Map, mcontext, Metadata, Metric, MetricMap, mflags, minor, mmsghdr, Moddi3, mode, Modsi3, Modti3, MonitorMsg, MOUNT, mprot, mq, mqd, msflags, msghdr, msginfo, msglen, msgqnum, msqid, Muldf3, Mulodi4, Mulosi4, Muloti4, Mulsf3, Multi3, Mut, Mutex, MutexGuard, MyCollection, n16, NamePadding, NativeLibBoilerplate, nfds, nl, nlink, NodeRef, NoneError, NonNull, NonZero, nthreads, NulError, OccupiedEntry, off, off64, oflags, Once, OnceState, OpenOptions, Option, Options, OptRes, Ordering, OsStr, OsString, Output, OVERLAPPED, Owned, Packet, PanicInfo, Param, ParseBoolError, ParseCharError, ParseError, ParseFloatError, ParseIntError, ParseResult, Part, passwd, Path, PathBuf, PCONDITION, PCONSOLE, Peekable, PeekMut, Permissions, PhantomData, pid, Pipes, PlaceBack, PlaceFront, PLARGE, PoisonError, pollfd, PopResult, port, Position, Powidf2, Powisf2, Prefix, PrefixComponent, PrintFormat, proc, Process, PROCESS, processentry, protoent, PSRWLOCK, pthread, ptr, ptrdiff, PVECTORED, Queue, radvisory, RandomState, Range, RangeFrom, RangeFull, RangeInclusive, RangeMut, RangeTo, RangeToInclusive, RawBucket, RawFd, RawHandle, RawPthread, RawSocket, RawTable, RawVec, Rc, ReadDir, Receiver, recv, RecvError, RecvTimeoutError, ReentrantMutex, ReentrantMutexGuard, Ref, RefCell, RefMut, REPARSE, Repeat, Result, Rev, Reverse, riflags, rights, rlim, rlim64, rlimit, rlimit64, roflags, Root, RSplit, RSplitMut, RSplitN, RSplitNMut, RUNTIME, rusage, RwLock, RWLock, RwLockReadGuard, RwLockWriteGuard, sa, SafeHash, Scan, sched, scope, sdflags, SearchResult, SearchStep, SECURITY, SeekFrom, segment, Select, SelectionResult, sem, sembuf, send, Sender, SendError, servent, sf, Shared, shmatt, shmid, ShortReader, ShouldPanic, Shutdown, siflags, sigaction, SigAction, sigevent, sighandler, siginfo, Sign, signal, signalfd, SignalToken, sigset, sigval, Sink, SipHasher, SipHasher13, SipHasher24, size, SIZE, Skip, SkipWhile, Slice, SmallBoolTrie, sockaddr, SOCKADDR, sockcred, Socket, SOCKET, SocketAddr, SocketAddrV4, SocketAddrV6, socklen, speed, Splice, Split, SplitMut, SplitN, SplitNMut, SplitPaths, SplitWhitespace, spwd, SRWLOCK, ssize, stack, STACKFRAME64, StartResult, STARTUPINFO, stat, Stat, stat64, statfs, statfs64, StaticKey, statvfs, StatVfs, statvfs64, Stderr, StderrLock, StderrTerminal, Stdin, StdinLock, Stdio, StdioPipes, Stdout, StdoutLock, StdoutTerminal, StepBy, String, StripPrefixError, StrSearcher, subclockflags, Subdf3, SubI128, SuboI128, SuboU128, subrwflags, subscription, Subsf3, SubU128, Summary, suseconds, SYMBOL, SYMBOLIC, SymmetricDifference, SyncSender, sysinfo, System, SystemTime, SystemTimeError, Take, TakeWhile, tcb, tcflag, TcpListener, TcpStream, TempDir, TermInfo, TerminfoTerminal, termios, termios2, TestDesc, TestDescAndFn, TestEvent, TestFn, TestName, TestOpts, TestResult, Thread, threadattr, threadentry, ThreadId, tid, time, time64, timespec, TimeSpec, timestamp, timeval, timeval32, timezone, tm, tms, ToLowercase, ToUppercase, TraitObject, TryFromIntError, TryFromSliceError, TryIter, TryLockError, TryLockResult, TryRecvError, TrySendError, TypeId, U64x2, ucontext, ucred, Udivdi3, Udivmoddi4, Udivmodsi4, Udivmodti4, Udivsi3, Udivti3, UdpSocket, uid, UINT, uint16, uint32, uint64, uint8, uintmax, uintptr, ulflags, ULONG, ULONGLONG, Umoddi3, Umodsi3, Umodti3, UnicodeVersion, Union, Unique, UnixDatagram, UnixListener, UnixStream, Unpacked, UnsafeCell, UNWIND, UpgradeResult, useconds, user, userdata, USHORT, Utf16Encoder, Utf8Error, Utf8Lossy, Utf8LossyChunk, Utf8LossyChunksIter, utimbuf, utmp, utmpx, utsname, uuid, VacantEntry, Values, ValuesMut, VarError, Variables, Vars, VarsOs, Vec, VecDeque, vm, Void, WaitTimeoutResult, WaitToken, wchar, WCHAR, Weak, whence, WIN32, WinConsole, Windows, WindowsEnvKey, winsize, WORD, Wrapping, wrlen, WSADATA, WSAPROTOCOL, WSAPROTOCOLCHAIN, Wtf8, Wtf8Buf, Wtf8CodePoints, xsw, xucred, Zip, zx}
%
, morekeywords=[5]{assert!, assert_eq!, assert_ne!, cfg!, column!, compile_error!, concat!, concat_idents!, debug_assert!, debug_assert_eq!, debug_assert_ne!, env!, eprint!, eprintln!, file!, format!, format_args!, include!, include_bytes!, include_str!, line!, module_path!, option_env!, panic!, print!, println!, select!, stringify!, thread_local!, try!, unimplemented!, unreachable!, vec!, write!, writeln!}  % prelude macros
}%

\lstdefinestyle{colouredRust}%
{ basicstyle=\ttfamily%
, identifierstyle=%
, commentstyle=\color[gray]{0.4}%
, stringstyle=\color[rgb]{0, 0, 0.5}%
, keywordstyle=\bfseries% reserved keywords
, keywordstyle=[2]\color[rgb]{0.75, 0, 0}% traits
, keywordstyle=[3]\color[rgb]{0, 0.5, 0}% primitive types
, keywordstyle=[4]\color[rgb]{0, 0.5, 0}% type and value constructors
, keywordstyle=[5]\color[rgb]{0, 0, 0.75}% macros
, columns=spaceflexible%
, keepspaces=true%
, showspaces=false%
, showtabs=false%
, showstringspaces=true%
}%

\lstdefinestyle{boxed}{
  style=colouredRust%
, numbers=left%
, firstnumber=auto%
, numberblanklines=true%
, frame=trbL%
, numberstyle=\tiny%
, frame=leftline%
, numbersep=7pt%
, framesep=5pt%
, framerule=10pt%
, xleftmargin=15pt%
, backgroundcolor=\color[gray]{0.97}%
, rulecolor=\color[gray]{0.90}%
}
\lstdefinestyle{c}{language=C}
\lstdefinestyle{cpp}{language=C++}
\lstdefinestyle{Java}{language=Java}
\lstdefinestyle{JavaScript}{language=JavaScript}
\lstdefinestyle{TypeScript}{language=JavaScript}
\lstdefinestyle{Python}{language=Python}
\lstdefinestyle{CSharp}{language=[Sharp]C}
\lstdefinestyle{PHP}{language=PHP}
\lstdefinestyle{Go}{language=Go}
\lstdefinestyle{Kotlin}{language=Kotlin}
\lstdefinestyle{Rust}{language=Rust}

% ======== Custom package setup end ========

\autor{Jacek Olczyk}{385896}

\title{Default values in Haskell record fields}
\titlepl{Domyślne wartości pól rekordów w Haskellu}

\kierunek{Computer Science}

\opiekun{PhD. Josef Svenningsson\\
  Chalmers University of Technology\\
}
\opiekunuw{PhD. Marcin Benke\\
  Instytut Informatyki\\
}

  % miesiąc i rok:
\date{Dec 2022}

%Podać dziedzinę wg klasyfikacji Socrates-Erasmus:
\dziedzina{ 
11.3 Informatyka\\ 
}

%Klasyfikacja tematyczna wedlug AMS (matematyka) lub ACM (informatyka)

% candidates sourced from https://dl.acm.org/ccs

% Software and its engineering
% Software notations and tools
% General programming languages
% Language types
% Functional languages

% Software and its engineering
% Software notations and tools
% General programming languages
% Language features
% Data types and structures

% Software and its engineering
% Software notations and tools
% Formal language definitions
% Syntax

\klasyfikacja{D. Software\\
  D.127 TODO pick candidate\\
  D.127.6. Numerical blabalysis}

% Słowa kluczowe:
\keywords{functional programming, Haskell, TODO}

% Tu jest dobre miejsce na Twoje własne makra i środowiska:

\newtheorem{defi}{Definition}[section]
\newcommand{\jcom}[1]{\textcolor{RoyalBlue}{[jo: #1]}}



% koniec definicji

\begin{document}
\maketitle

\begin{abstract}
  This thesis presents a prototype implementation of the syntax
  and semantics for specifying the default values for fields 
  of record types in Haskell. The proposed syntax aims to have
  no impact on existing Haskell code and minimal syntactic intrusion
  when the proposed feature is used. The use of default values
  can and will happen only when using the braced record construction
  syntax already present in Haskell. The syntax for defining default 
  values for fields follows recognizable conventions from other
  programming languages in which this feature is present.
  This feature facilitates more seamless integration of Haskell codebases 
  into multi-language codebases unified
  by interface definition languages such as Apache Thrift. 
\end{abstract}

\tableofcontents
%\listoffigures
% \listoftables

\chapter*{Introduction}
\addcontentsline{toc}{chapter}{Introduction}

Algebraic data types have been a staple of functional programming languages since their earliest days.
A typical extension of their utility for everyday programming is the record syntax, present in most popular functional programming languages.
However, a notable functionality often missing from that feature is the ability to provide default values for individual record fields.
While widely used workarounds are known, they suffer from various problems that a compiler-assisted solution can avoid.

This work aims to solve these problems in the case of the Haskell language \cite{HudakHaskell}
by proposing a custom syntax inside record definitions that enables the programmer to provide a default value to any field of a given record type.
In the beginning, we further detail the background and motivation behind these changes, as well as examples of problems they solve.
Then, we outline the syntactic changes applied to the grammar of Haskell alongside alternative solutions and reasoning behind the final choice.
Following that, we explain in detail the semantic behavior of introduced features and the purely semantic changes to existing constructs.
We also explain the development process and decisions made when implementing the feature in GHC, the leading compiler of the Haskell language.


\chapter{Background}\label{r:bg}
\section{Overview of record syntax and field defaulting}
A major drawback algebraic data types usually face when contrasted with data types commonly found in imperative languages (like \texttt{struct} in C) is their unwieldiness when it comes to data structures containing many fields.
While most imperative and virtually all object-oriented languages feature a prominent and easy-to-use syntax for both naming and accessing individual fields of their data structures, the default for most functional languages has been to prefer data structures with unnamed fields.
However, as projects grow larger and larger, having only unnamed data structure fields becomes increasingly cumbersome and error-prone.
Thus, most functional languages provide an alternate record syntax for defining their data types. 

In most cases, record syntax provides two important features: 
the ability to name individual fields of a data structure; 
and the automatic generation of field selector functions which
absolves programmers from having to write the otherwise necessary,
but cumbersome boilerplate code.

As an example, let's compare record syntax in C and Haskell representing a person's name and age.

\begin{lstlisting}[style=c]
  typedef struct {
    char* name;
    int age;
  } Person;
\end{lstlisting}

\begin{lstlisting}
  data Person 
    = Person 
    { name :: String
    , age :: Int
    }
\end{lstlisting}
In C, without this syntax, the only way to create compound data structures consisting of data of different types is manual pointer arithmetic:\\
\begin{minipage}{\linewidth}
\begin{lstlisting}[style=c]
  // create a record with this syntax and set fields
  Person person;
  person.name = "Adam";
  person.age = 21;

  // one way to recreate the above on Linux without struct syntax
  void* person = alloca(sizeof(char*) + sizeof(int));
  *(char**)person = "Adam";
  *(int*)(person + sizeof(char*)) = 21;
  // the above would of course be extracted to accessor functions
  // in hypothetical real-world code
\end{lstlisting}
\end{minipage}

In Haskell, the lack of named fields would necessitate boilerplate accessor functions as described in \cite{WikibooksHaskellDatatypes}.

\subsection{History of records and field defaulting in other languages}
The earliest examples of programming language syntax for record types date back to the 1950s and 60s with their introduction in COBOL in 1959 \cite{sebesta1996concepts}.
Popular contemporary languages like Algol 68 \cite{van2012revised} and FORTRAN 77 \cite{fortran77} adopted their own syntax soon thereafter.
They did not provide any facilities for providing default values for structure fields.
Early object-oriented languages either used constructors for this purpose (Smalltalk \cite{smalltalk80}), or nothing at all (Simula \cite{Simula67}).
The same followed for most of languages developed in the 1970s and 80s: languages with either no default initialization of record types, like C and Pascal,
or default initialization via constructor functions, like in C++.
A notable exception is Common Lisp, which being released in 1985 can be counted 
as one of the earliest examples of records with dedicated syntax for default field values \cite{CommonLisp}.
With the rise of popularity of programming and the coming of the Internet era, 
syntactic support for default values has suddenly changed from obscure to ubiquitous, as well see from the examples in the next section.
\subsection{Modern examples in other languages}
Examining 10 of the most popular general-purpose programming languages in 2022,
most of them provide in their latest version some utility for specifying default field values,
either through constructor functions or dedicated syntax.
Chosen programming languages are: C++, Java, JavaScript, TypeScript, Python, C\#, PHP, Go, Kotlin and Rust.
Choice was based on statistics from \cite{StatistaProgLangs}, after removing markup languages, 
query languages and scripting languages, as well as C which was already discussed above.
Of course, any programming language offers support for default values through creating an object and using it as a "default".
This approach is not investigated, as it requires no help from the language to implement, unless there is a language-wide
convention on the details on how the object is supposed to be created. 


The below table illustrates support for field defaulting in the above languages.
\begin{table}[H]
  \centering
  \begin{tabular}{c|c|c|}
  \cline{2-3}
                                    & \begin{tabular}[c]{@{}c@{}}Dedicated syntax\end{tabular} & \begin{tabular}[c]{@{}c@{}}Parameterless constructors\end{tabular}  \\ \hline
  \multicolumn{1}{|c|}{C++}        & Since C++11                                                 & Yes                                                                   \\ \hline
  \multicolumn{1}{|c|}{Java}       & Yes                                                         & Yes                                                                   \\ \hline
  \multicolumn{1}{|c|}{JavaScript} & Since ES6                                                   & \begin{tabular}[c]{@{}c@{}}Since ES6, or as prototypes\end{tabular} \\ \hline
  \multicolumn{1}{|c|}{TypeScript} & Yes                                                         & Yes                                                                   \\ \hline
  \multicolumn{1}{|c|}{Python}     & \begin{tabular}[c]{@{}c@{}}dataclasses only\end{tabular}  & Yes                                                                   \\ \hline
  \multicolumn{1}{|c|}{C\#}        & Yes                                                         & Yes                                                                   \\ \hline
  \multicolumn{1}{|c|}{PHP}        & Since PHP5                                                  & Since PHP5                                                            \\ \hline
  \multicolumn{1}{|c|}{Go}         & No                                                          & No                                                                    \\ \hline
  \multicolumn{1}{|c|}{Kotlin}     & Yes                                                         & Yes                                                                   \\ \hline
  \multicolumn{1}{|c|}{Rust}       & No                                                          & Limited, through the Default trait                                             \\ \hline
  \end{tabular}
  \caption{Support for specifying default field values across popular languages.}
  \label{fig:langtbl}    
\end{table}
Below are examples of each syntactic construct mentioned in the table. 
Note that, of course, having parameterless constructor syntax also allows for defaulting values
of an arbitrary subset of fields through adding parameters to the constructor.
\begin{itemize}
  \item C++: dedicated syntax (C++11 onwards) \cite{Cpp11}
  \begin{lstlisting}[style=cpp]
    struct Person {
      std::string name = "Adam";
      int age = 21;
    };
  \end{lstlisting}
  The above values get assigned to fields if the fields are missing in the 
  constructor's member initialization list (showcased in the next bullet).
  \item C++: default constructor \cite{Cpp98}
  \begin{lstlisting}[style=cpp]
    struct Person {
      std::string name;
      int age;
      Person() : name("Adam"), age(21) {}
    };
  \end{lstlisting}
  This approach uses the constructor's member initialization list, which guarantees that the fields are initialized
  with the default values before entering the constructor body.
  \item Java: dedicated syntax
  \begin{lstlisting}[style=Java]
    class Person {
      public String name = "Adam";
      public int age = 21;
    }
  \end{lstlisting}
  This approach assures that the default values are assigned before entering the body of any user-defined constructor.
  \item Java: parameterless constructor
  \begin{lstlisting}[style=Java]
    class Person {
      public String name;
      public int age;
      public Person() {
        name = "Adam";
        age = 21;
      }
    }
  \end{lstlisting}
  This approach is just manual assignment of values, the only syntactic help is the constructor itself.
  \item JavaScript: dedicated syntax (since ES6)
  \begin{lstlisting}[style=JavaScript]
    class Person {
      name = "Adam";
      age = 21;
    }
  \end{lstlisting}
  \item JavaScript: parameterless constructor (since ES6)
  \begin{lstlisting}[style=JavaScript]
    class Person {
      constructor() {
        this.name = "Adam";
        this.age = 21;
      }
    }
  \end{lstlisting}
  \item JavaScript: prototypes
  \begin{lstlisting}[style=JavaScript]
    function Person() {
      this.name = "Adam";
      this.age = 21;
    }
  \end{lstlisting}
  This defines a constructor function for \texttt{Person}, and method are added via prototypes.
  \item TypeScript: dedicated syntax
  \begin{lstlisting}[style=TypeScript]
    class Person {
      name = "Adam";
      age = 21;
    }
  \end{lstlisting}
  Note that the type of the fields is automatically deduced from the initializing expression.
  \item TypeScript: parameterless constructor
  \begin{lstlisting}[style=TypeScript]
    class Person {
      name: string;
      age: number;
      constructor() {
        this.name = "Adam";
        this.age = 21;
      }
    }
  \end{lstlisting}
  This approach is just manual assignment of values, the only syntactic help is the constructor itself.
  \item Python: dataclasses
  \begin{lstlisting}[style=Python]
    @dataclass
    class Person:
      name: str = "Adam"
      age: int = 21
  \end{lstlisting}
  \item Python: parameterless constructor
  \begin{lstlisting}[style=Python]
    class Person:
      def __init__(self):
        self.name = "Adam"
        self.age = 21
  \end{lstlisting}
  \item C\#: dedicated syntax
  \begin{lstlisting}[style=CSharp]
    class Person {
      public String Name = "Adam";
      public int Age = 21;
    }
  \end{lstlisting}
  This approach assures that the default values are assigned before entering the body of any user-defined constructor.
  \item C\#: parameterless constructor
  \begin{lstlisting}[style=CSharp]
    class Person {
      public String Name;
      public int Age;
      public Person() {
        Name = "Adam";
        Age = 21;
      }
    }
  \end{lstlisting}
  This approach is just manual assignment of values, the only syntactic help is the constructor itself.
  \item PHP: dedicated syntax (since PHP 5)
  \begin{lstlisting}[style=PHP]
    class Person {
      public $name = "Adam";
      public $age = 21;
    }
  \end{lstlisting}
  \item  PHP: parameterless constructor (since PHP 5)
  \begin{lstlisting}[style=PHP]
    class Person {
      public $name;
      public $age;
      function __construct() {
        $this->name = "Adam";
        $this->age = 21;
      }
    }
  \end{lstlisting}
  Note that this is not actually custom syntax, constructors in PHP are regular methods
  with the name \texttt{\_\_construct}.
  \item Kotlin: dedicated syntax and parameterless constructor
  \begin{lstlisting}[style=Kotlin]
    class Person {
      val name = "Adam";
      val age = 21;
    }
  \end{lstlisting}
  The body of the class is explicitly considered to be the primary constructor in Kotlin.
  \item Kotlin: parameterless (secondary) constructor
  \begin{lstlisting}[style=Kotlin]
    class Person {
      val name: String;
      val age: Int;
      constructor() {
        name = "Adam";
        age = 21;
      }
    }
  \end{lstlisting}
  \item Rust: Default trait
  \begin{lstlisting}[style=Rust]
    struct Person {
      name: String;
      age: i32;
    }

    impl Default for Person {
      fn default() -> Person {
        Person {
          name: "Adam",
          age: 21,
        }
      }
    }
  \end{lstlisting}
  This approach is limited by only being able to provide default values to all fields 
  of the struct at once.
\end{itemize}

\subsection{Conclusion}
From the above considerations it is clear that the vast majority of modern languages 
offer a convenient syntax for providing default field values. 
Comparing that to the historical situation where no major language offered it,
it is clear that it has become a quality-of-life feature 
that today's programmers have come to expect from their language of choice.
Thus, such an addition is very likely to positively impact the software development
experience for Haskell programmers.

\section{Field defaulting in Haskell}
\subsection{Introduction}
Despite record syntax's relative ubiquity in modern functional programming languages, most implementations
of it (e.g. OCaml, F\#, ReasonML) have an important usability feature missing when compared
to their imperative counterparts --- the ability to leave out certain fields empty
when creating a concrete object of a given type. Especially when dealing with
data structures containing a great number of fields, e.g. types that represent
a configuration file, the necessity of providing every field 
of the structure with a value can become unwieldy very quickly. 
The only example of this syntax in a popular functional programming language seems to be Erlang,
where records can be default-constructed and the fields without default values are undefined \cite{ErlangRecords}.

\subsection{Prior attempts at solving the issue}
This issue has been heretofore partially mitigated thanks to another common feature of record syntax in programming languages: the record update syntax. 
It allows for the construction of a record in such a way that any unspecified fields are copied over from an already existing record. 
If the author of a type then provides their users with a globally visible ''default'' object with all its fields already filled out, a user can utilize the record update syntax to mimic the desired functionality.

However, this approach is not without its drawbacks.
Firstly, it can prove a challenge if some fields don't lend themselves easily to default values. 
For example, a data type might have an ID component along with invariants guaranteeing its uniqueness. 
In such a situation, great care must be taken when creating a new record to ensure that the ID component is always modified from the value provided by default.
It is also possible to circumvent that using optional types (like Haskell's \texttt{Maybe}), but this makes the data structure cumbersome to use for anything other than its creation.
Unlike it is for its imperative counterpart, this approach does not offer the programmer any automated systems that would detect the absence of such a value.
Given that a major advantage functional languages tout over others is the strength of their automated compile-time error detection, this is an area in which clear improvement can be made.

In some languages, like Haskell, it is also possible to create a record without specifying all of its fields.
However, in that case, the fields are still automatically filled with bottom values.
To make matters worse, such a creation creates compile-time warnings when defining the default record, and no compile-time warnings when some of the bottoms are not replaced.
Because of that, this solution is even worse than the previous one.

Secondly, the necessity of creating a custom object to use as a default can lend itself to improper coding style and more difficulty in codebase navigation.
In order to create a default object, one must first decide on a name that it will be given, which in large codebases necessitates either coming up with a naming convention or dealing with the repercussions of inconsistent naming. 
Furthermore, there is no requirement for the programmers to place the default object in the same place, or even in the same file as the definition of the type itself.
This can quickly make the issue of finding the default object a true detective's task. 

\subsection{Direct use cases}
Aside from the obvious functionality improvements, allowing default fields in records has positive impact on the overall Haskell ecosystem.
Many large multi-language codebases, such as the giant monorepos used by Big Tech companies, use interface definition languages
(IDLs) like Apache Thrift or Google's Protobuf to interface between code written in different languages.
IDLs usually provide this through a language-agnostic syntax for defining data structures. 
This syntax usually (as is the case in Apache Thrift) contains a way to provide default values for fields of the data structures being defined.
Because of that, any language that doesn't natively support default field values in data structures will need to use workarounds before being able to interface with IDLs.

\subsubsection{Code migrations}
\newcommand{\user}{\textit{user}}
\newcommand{\infra}{\textit{infra}}
A major potential real-world use case is helping with code migrations.
As an example, consider a system with two codebases, \user{} and \infra{}. 
\infra{} provides type definitions for configs, like this simplified example:\\
\begin{minipage}{\linewidth}
\begin{lstlisting}
  data Config 
    = Config
    { name :: String
    , value :: SomeType
    }
\end{lstlisting}
\end{minipage}
Code in \user{} will then construct values of this type, like so:\\
\begin{minipage}{\linewidth}
\begin{lstlisting}
  someVar :: Config
  someVar = Config { name = "my config", value = myValue }
\end{lstlisting}
\end{minipage}
To ensure runtime safety of the system, constructing this value is protected via the combination of the 
\texttt{-Wmissing-fields} and \texttt{-Werror} compiler flags. 
In their presence, omitting a field creates a compile-time error, making sure that every field value is correctly in place.
Since it is easy for a programmer to forget about a field, especially when there are many of them, this check
measurably increases correctness of the system. 
However, when there is a need to add an additional field, problems with that approach arise. 
Even if the system was limited to just one codebase, a sufficiently large system cannot be easily migrated to use a new value
in all the occurrences of the construction. With multiple codebases, the process complicates even more. Here's a step-by-step
example of how that could look like, if one wanted to add a field \texttt{owner} of type \texttt{String}:
\newcommand{\defaultconf}{\texttt{defaultConfig}}
\newcommand{\conf}{\texttt{Config}}

\begin{enumerate}
  \item We cannot add the field directly without creating compile errors in \user{} code. 
  To work around that, we create a default value called, let's say, \texttt{defaultConfig} in the same module of \infra{} code as 
  the definition of the type.\\
  \begin{minipage}{\linewidth}
    \begin{lstlisting}
defaultConfig :: Config
defaultConfig = Config { name = "default name", value = defaultSomeTypeValue }
    \end{lstlisting}
  \end{minipage}
  \item This code needs to be released in order for \defaultconf{} to be accessible in \user{} code.
  \item Then, all constructors of \conf{} in \user{} can be modified to a record update of \\
  \defaultconf{}:\\
  \begin{minipage}{\linewidth}
    \begin{lstlisting}
someVar :: Config
someVar = defaultConfig { name = "my config", value = myValue }
    \end{lstlisting}
  \end{minipage}
  This change silences the warning 
  \item This code, again, needs to be released before any further changes in \infra{} are made.
  \item With that setup, we can safely introduce a new field in \infra{}, immediately modifying \defaultconf{} with a default value :\\
  \begin{minipage}{\linewidth}
    \begin{lstlisting}
data Config 
  = Config
  { name :: String
  , value :: SomeType
  , owner :: String
  }

defaultConfig :: Config
defaultConfig 
  = Config 
  { name = "default name"
  , value = defaultSomeTypeValue
  , owner = "default owner" 
  }
    \end{lstlisting}
    \end{minipage}
  \item Yet again, a release is needed.
  \item Now, all of the \user{} code can revert to using constructors, as soon as the relevant code is adapted to make use of the 
  new field. \\
\begin{minipage}{\linewidth}
\begin{lstlisting}
someVar :: Config
someVar 
  = Config
  { name = "my config"
  , value = myValue
  , owner = "my owner"
  }
\end{lstlisting}
\end{minipage}

  \item This \user{} code must again be released. This restores the protection given by \\
  \texttt{-Wmissing-fields}.
\end{enumerate}

By contrast, with the addition of field default value syntax the above process could be shortened 
to just one \infra{} release containing the new field with the default value:\\
\begin{minipage}{\linewidth}
\begin{lstlisting}
data Config 
  = Config
  { name :: String
  , value :: SomeType
  , owner = "default owner" :: String
  }
\end{lstlisting}
\end{minipage}
This way, \texttt{-Wmissing-fields} is working for all the other fields, and teams working on \user{}
code can add support for the new field at their own pace.
Once this is done, the default value can be removed to ensure it's also protected by \texttt{-Wmissing-fields}.
\chapter{GHC architecture}
This section contains an abridged overview of the architecture of the frontend of the GHC compiler to serve as background for implementation details provided later.
The compilation process in the GHC compiler consists of 6 main phases \cite{GHCIntermediateForms}, the first 4 of which are:
\begin{enumerate}
  \item Parsing
  \item Renaming
  \item Typechecking 
  \item Desugaring 
\end{enumerate}
These are all the phases that deal with the Haskell AST, mostly unchanged from the parser. 
The last one, the desugarer, desugars the Haskell AST to the much smaller internal language Core, which is used for optimization.
Before discussing the passes themselves, it's worth to describe how GHC deals with modification of the AST throughout the frontend of the compiler.
\section{Trees that grow}\label{ttg}
Throughout the phases of the frontend, various data about the Haskell AST is generated (and other becomes no longer needed).
In order to avoid data structure fields or constructors being unused during some phases or having multiple slightly different versions of the same AST type definitions,
a mechanism was introduced in \cite{Najd2017TreesTG} which parameterized the entire syntax tree with a type argument representing the current phase of compilation.
Three distinct phases are available: \texttt{GhcPs}, \texttt{GhcRn}, \texttt{GhcTc},
with the parser producing code parameterized by \texttt{GhcPs}, the renamer converting \texttt{GhcPs} to \texttt{GhcRn},
and the typechecker converting \texttt{GhcRn} to \texttt{GhcTc}. 
With that in mind, when creating a data structure,
the developer writes all constructors and fields common to all phases as they normally would, with a minor exception.
Every type \texttt{T} gets an extra constructor \texttt{XT} (e\texttt{X}tension to \texttt{T}), which allows for phase-specific constructors,
and every constructor \texttt{CtorT} has as its first field set to type \texttt{XCtorT}, to allow for phase-specific constructor fields.
Any type written in extension field or an extension constructor definition is actually a \texttt{type family},
which when parameterized by the phase, allows for a different \texttt{type instance} for each phase.
\section{Compilation phases}
\subsection{Parsing}
GHC does parsing of source code through a monolithic Happy grammar \cite{Happy}, although this wasn't initially the case \cite{Jones1993TheGH}.
The data structures emitted by the parser are often ambiguous as to which production was actually used to parse a given part of the source code.
This is a result of a policy of ''overparse, then filter out the bad cases'' \cite{ParserWiki}.
For example, in certain contexts both patterns and expressions are valid, so knowing which grammar derivation to use would require infinite lookahead.
This is solved through a validation monad which allows parsing contexts to specify which constructs are possible, failing the parse on the rest.
Another system of note are the source location and annotation types, which are ingrained into the types that form the Haskell source code AST.
Any parser node has to extract the location and annotation data from its subnodes, and very carefully combine them together,
as there is no automatic mechanism for detecting errors in code like this.

\subsection{Renaming}
The task of the renamer is to match each name usage in the code with the matching definition. 
Its code is intertwined with the typechecker code, and uses the same monad \cite{RenamerWiki}.

\subsection{Typechecking}
Typechecking a module happens concurrently (but not in parallel) with renaming, because of Template Haskell splices.
The typechecker uses the \texttt{Type} type, which is used to type Core, instead of the \texttt{HsType} created by the parser \cite{TypecheckerWiki}.
This is because the typechecking process is quite complicated, and thus requires a data structure that is easy to manipulate.

\subsection{Desugaring}
The desugarer transforms the Haskell AST (with the \texttt{GhcTc} phase indicator) into an explicitly type variant of System FC called Core \cite{CoreWiki}.
The Core language AST consists of very few constructors, and great care is taken to not modify this language when not necessary, 
so any changes that are little more than syntactic sugar should be implemented as part of the desugarer.

\chapter{Syntax}
Among a wide variety of languages supporting providing default values to fields of data structures, there has been a virtually unanimous agreement on the syntax used for providing the default values.
Any language that supports this feature on the compiler level seems to agree on the base syntax being \texttt{field\_name = value}, with minor variations depending on the given language's syntax for field type annotations.
Thus, declaring an integer field with default value of 1 in C++, Java, and likely many others, looks like this: \texttt{\textcolor{Violet}{int} x = \textcolor{BrickRed}{1};}.
As another example, the same is accomplished in TypeScript with the syntax \texttt{x:\ \textcolor{Violet}{number} = \textcolor{BrickRed}{1},}.

\section{Syntax for declaring a default field value}
With that in mind, we needed to accommodate the \texttt{field\_name = value} syntax to existing syntax for Haskell records.
A single field in a Haskell record currently looks like the following:
\begin{lstlisting}[escapeinside={(*}{*)}]
  fieldName :: FieldType
\end{lstlisting}
Considering the above examples, a few alternative approaches for the default value syntax come to mind.

\begin{itemize}
  \item type after value
  \begin{lstlisting}[escapeinside={(*}{*)}]
    fieldName = field_value :: FieldType
  \end{lstlisting}
  \item type after name
  \begin{lstlisting}[escapeinside={(*}{*)}]
    fieldName :: FieldType = field_value
  \end{lstlisting}
  \item variants of both of the above, but allowing for omitting the type of the expression
\end{itemize}

In total four variants, and the one chosen here is the "type after value" approach without omitting the type signature.
\subsection{Order of syntactic components}
To decide between ''type after value'' and ''value after type'', we looked at language constructs already prevalent in existing Haskell code.

%While it is possible for a top-level Haskell binding to be declared both of these ways, the usual preference for Haskell developers is 
A regular top-level Haskell binding with a type annotation can have multiple forms:
\begin{enumerate}
  \item \label{vb:popular} standalone type annotation
  \begin{lstlisting}[escapeinside={(*}{*)}]
    name :: Type 
    name = value
  \end{lstlisting}
  \item \label{vb:good} type after value
  \begin{lstlisting}[escapeinside={(*}{*)}]
    name = value :: Type
  \end{lstlisting}  
  \item \label{vb:bad} value after type
  \begin{lstlisting}[escapeinside={(*}{*)}]
    name :: Type = value
  \end{lstlisting}
\end{enumerate}

By far, the most popular of these variants is \ref*{vb:popular}, but it's not readily adaptable to record fields.
To choose between \ref*{vb:good} and \ref*{vb:bad}, we note that \ref*{vb:bad} is very rarely used in real code.
Furthermore, at first glance, it might seem as if the default value is being assigned to the type, not the field name.
In contrast, \ref*{vb:good} looks like a familiar construct --- a binding name without a type annotation, and its right-hand-side expression with one (and it is already parsed as such by GHC).

\subsection{Omitting the type signature}
It would seem like a good idea to allow users to omit the type signature for a field with a user-supplied default value.
After all, it seems possible to always infer the type from the defaulting expression.
As it turns out, the reality is not that simple.
Many tools that are part of the GHC project currently need to be able to present the type of all fields of a given datatype without the information provided by a full typecheck of the program.
The list of these tools includes the typechecker itself.
While the typechecker itself could theoretically be modified to avoid this problem, forcing a typecheck on other tools (such as Haddock) would unnecessarily slow down their performance without a noticeable gain.
As providing type annotations is a good practice in almost all use cases and virtually all existing code already has to do so for all their record fields, this loss of functionality seems to have negligible cost.


\section{Syntax for use of defaulted fields}

The chosen method for initializing fields with their default value is through braced record syntax.
Because the only other way to construct a record is through a regular constructor function, which cannot have optional arguments (just like all other Haskell functions), braced record syntax is the sole method of assigning default values to fields.
Omitting a given defaulted field is the only requirement for activating the mechanism. 
Since omitting fields in record construction was already allowed (but produced a compile-time warning), no syntax change is necessary.

\section{Summary}
Below is a complete example of a Haskell record type with one of its fields defaulted. 
In our view, the final syntax combines the best compromise between readability, new user learning curve, and implementation viability.

\begin{lstlisting}[escapeinside={(*}{*)}]
data NewRecordType
   = NewRecordTypeCtor 
   { fieldName :: SomeType 
   , fieldName2 = expression2 :: SomeType2
   }
\end{lstlisting}
No syntax change is necessary for the case of the usage of default values.


\chapter{Semantics}
\section{Examples}
The semantics for the proposed extension of the language are very simple.
Recall the (slightly renamed) example type from the previous chapter:

\begin{lstlisting}[escapeinside={(*}{*)}]
  data Record1
     = RecordCtor2 
     { field1 :: Int
     , field2 = expression1 :: String
     }
\end{lstlisting}
  
In order to utilize the mechanism inserting default values, usage of the record construction syntax is necessary:

\begin{lstlisting}[escapeinside={(*}{*)}]
  exampleRecord :: Record1
  exampleRecord 
    = RecordCtor1
    { field1 = 12
    -- field2 is initialized with the default value
    }
\end{lstlisting}

Notably, this is the only possible way of constructing a record with a non-$\bot$ default value.
Below we give non-examples of constructions that do not change their semantics with this change (but could reasonably be expected to).
In the first example, the field is initialized with the $\bot$ value, both in the current version of Haskell and with the changes described in this thesis applied to GHC.

\begin{lstlisting}[escapeinside={(*}{*)}]
  data Record2
     = RecordCtor2 
     { field1 :: Int
     , field2 :: String
     }  

  exampleRecord :: Record2
  exampleRecord 
    = RecordCtor2 
    { field1 = 12
    -- field2 is initialized with the (*$\bot$*) value, raising an error when evaluated
    }

\end{lstlisting}

In the second example, we note that record update is not affected by these changes. 
Every unspecified field in the record update will be copied directly from the old record, even if its value is $\bot$.
The example record created by the following code is equal to \texttt{RecordCtor3 \{ field1 = 5, field2 = 4 \}}.

\begin{lstlisting}[escapeinside={(*}{*)}]
  data Record3
     = RecordCtor3
     { field1 = 1 :: Int
     , field2 = 2 :: String
     }  

  baseRecord :: Record3
  baseRecord = RecordCtor3 { field1 = 3, field2 = 4 }

  exampleRecord :: Record3
  exampleRecord 
    = baseRecord
    { field1 = 5
    -- field2 is initialized using the value in baseRecord, not the default value
    }

\end{lstlisting}

\section{A more formal statement}
For precision, we offer a reasonably rigorous semantic description of field defaulting:
\newtheorem{remark}{Description}
\begin{remark}
  \normalfont
  Let $T$ be a type with a record constructor $T'$ containing a field $f$ for which a default expression $e$ has been provided.
  Then, if while calling the constructor $T'$ via the record construction syntax, field $f$ is not given any value, during evaluation the construction happens \textit{as-if} field $f$ had been specified and given the value $e$.
\end{remark}

\section{Notable observations}
Given Haskell's usual semantics as a pure and lazy language, the defaulting expression is only evaluated at most once, if it's needed (more details in \ref{sec:efficiency}).
This allows for arbitrarily complex expressions without risking unnecessarily long runtime of the program. 
Even if the defaulted field's type is a computation inside the \texttt{IO} monad (and thus can be considered to have side effects), the evaluation of the expression only produces instructions for the runtime system without actually executing them.
Thanks to that we can be certain that the default values will behave as expected.


\chapter{Implementation}

We implemented the proposed change to the language inside the leading Haskell compiler, GHC (Glorious Haskell Compiler) \cite{HudakHaskell}.
The algorithm is a journey that takes the default expression from being declared to being substituted when a field is missing.
It consists of the following main steps:
\begin{enumerate}
  \item Parse the defaulting expression inside the constructor.
  \item Give it a unique name and group it together with top-level bindings so that dependency analysis, renaming and typechecking are performed without disturbances.
  \item Mark defaulted fields as not actually missing.
  \item During desugaring to Core, replace any missing fields with their default values.
\end{enumerate}

The necessary changes made to the GHC code can be broken down into X parts:
\begin{enumerate}
  \item Changes to datatypes
  \item Parser
  \item Renamer
  \item Desugarer
\end{enumerate}

\section{Changes to datatypes} 

For the initial prototype we tried to keep the changes to existing internal GHC data structures to a minimum.
First necessary change was to the \texttt{ConDeclField} type, which represents a single field of a record constructor.
We modify it by adding a field called \texttt{cd\_fld\_ini} of type \texttt{Maybe (LHsExpr pass)}. 
The \texttt{L} prefix in \texttt{LHsExpr} means that the expression contains information about its location in the source file.
The \texttt{pass} parameter signifies the current phase of the compilation (using the Trees that grow technique described in \ref{ttg} and in detail in \cite{Najd2017TreesTG}).
This modification allows us to pass the defaulting expression from the parser to the renamer.

The renamer then moves information about types and their constructors from the data structures that represent abstract syntax to internal structures that are easier to work with.
There, information about record fields is stored inside the \texttt{FieldLabel} type, to which we added a new field, \texttt{flIniExpr :: Maybe Name}.
The \texttt{Name} refers to the generated name of the binding generated for the expression.

\section{Parser}
Parser uses Happy, a grammar-based parser generator for Haskell.
The existing grammar node for record fields is called \texttt{fielddecl}.
Its only production is \texttt{fielddecl : sig\_vars '::' ctype}.
The above parses a comma-separated list of variable names, followed by the \texttt{::} operator, followed by the type of the field(s).
The names are a list to allow for declaring multiple fields with the same type in one line, e.g. \texttt{a, b, c :: Foo}.
We want to modify it to allow for \texttt{ = expr} between the name of the field and the type.
One obvious solution would be to add another production producing \texttt{var '=' exp '::' ctype} (as we only want to allow a single name to be defaulted on one line).
This unfortunately produces a reduce/reduce conflict, as \texttt{exp} can already contain a type signature. 
We solve this conflict by extracting the production for an expression with a type signature to a separate node, \texttt{typedexp}.
This allows us to parse \texttt{var '=' typedexp}, which forces a type signature and guarantees no conflicts.
The aforementioned production is introduced as a new node, \texttt{varini}, for clarity.

\section{Renamer}
\subsection{Preprocessing}
Before we start the renaming process, we want to make sure we are making as much use of existing GHC piping as possible.
Thus, at the beginning of the renamer, we extract all of the default values for fields into separate bindings.
In the type declarations, the user-provided expressions are then replaced (inside the \texttt{ConDeclField} structures) with simple variable expressions.
The user-provided expressions are given freshly generated names and grouped together with other value declarations for the module.
All the declarations are fed to the renamer in this way, and there the next phase of renaming proceeds.
\subsection{Conversion}
Early in the renaming process, the abstract syntax for type declarations is parsed into more convienient data structures.
It is here that record selector functions are created and their names are placed inside the \texttt{FieldLabel} objects created alongside them.
This means that it's the perfect place for inserting the names of the default expressions into the \texttt{FieldLabel}s.
To do that, before checking individual constructors or fields, a mapping is constructed of all field names from a given type declaration to the names of their default expressions.
\subsection{Missing fields detection}
The fields that have been given default values should be excluded from both the list of missing strict fields (which produces an error when nonempty) and the list of missin nonstrict fields (which only produces a warning).
When detecting if fields are missing, the algorithm compares the available \texttt{FieldLabel}s for the given constructor with the occurrences provided in a given expression.
Thus, to exclude defaulted fields, it is sufficient to filter out the \texttt{FieldLabel}s that have their \texttt{flIniExpr} field set to a \texttt{Just} value.

\section{Desugarer}
During desugaring, we need to fill any unspecified record fields with the variable name corresponding to the appropriate default expression.
Normally, the desugarer fills all missing fields with error ($\bot$) values.
Given that the desugarer has access to the \texttt{FieldLabel}s of the arguments being desugared, we can check if they contain the default expression and desugar the field to contain the variable name instead of an error.

\section{Notes on global bindings}
\label{sec:efficiency}
A notable side effect of the decision to implement default values as global bindings is that it implicitly performs the optimization
described in \cite{partain1996let-floating} as the full laziness transformation. 
This naturally is expected to result in the same consequences and considerations as outlined therein:
\begin{itemize}
  \item Programs can be made faster by not repeating work (to compute the value) and allocations (to store the value).
  \item When the default value ''is already a value, or reduces to a value with a negligible amount of work'', the only potential gain is in allocations.
  \item The values can potentially cause a space leak 
    (e.g. when a default value is an infinite list and more elements get evaluated than will be needed in the future)
  \item They are harder to garbage-collect
\end{itemize}
How do these apply to the typical use case of default field values?
In a reasonable program, one would expect default values to be more likely to need to be reused.
Thus, making sure that their value isn't recomputed on each evaluation of a constructor that does not contain the field seems like a good approach.
It also follows the principle of least confusion \cite{saltzer2009principles}: 
it's not immediately obvious to a user writing a constructor with a defaulted field how much work will be performed on construction,
so the best approach is to minimize it.

Furthermore, it's common for default values to require little to no computation. 
For example, a major use case of default values is placeholders meant to be filled out later. 
Often programmers choose the simplest possible value for a given type as the default, e.g. \texttt{0\ ::\ Int}.
With this in mind, it seems unlikely for programmers to provide default values that could cause a noticable space leak, and most of the time one might
reasonably expect a bunch of allocations to be omitted.
Unfortunately, there is no way to verify these assumptions without developing an alternative implementation that doesn't make the default values global.

\chapter{Conclusion} 
This work proposed a custom syntax for default record fields in Haskell, detailed a proof-of-concept implementation and discussed the design choices
and tradeoffs behind the syntax and semantics of the implementation.

From this work it is clear that an implementation of default record field values in Haskell is possible, usable useful and hopefully soon implemented
into the mainline GHC. The syntax is clear and intuitive and the behavior of programs that use it is predictable and efficient.
The next step is to submit a proposal for the syntax to be accepted as a language extension in the GHC compiler, to bring this feature to the widest possible audience.

\bibliographystyle{acm} \bibliography{bibliography}

\end{document}


%%% Local Variables:
%%% mode: latex
%%% TeX-master: t
%%% coding: latin-2
%%% End:
